% Options for packages loaded elsewhere
\PassOptionsToPackage{unicode}{hyperref}
\PassOptionsToPackage{hyphens}{url}
%
\documentclass[
]{article}
\usepackage{amsmath,amssymb}
\usepackage{lmodern}
\usepackage{iftex}
\ifPDFTeX
  \usepackage[T1]{fontenc}
  \usepackage[utf8]{inputenc}
  \usepackage{textcomp} % provide euro and other symbols
\else % if luatex or xetex
  \usepackage{unicode-math}
  \defaultfontfeatures{Scale=MatchLowercase}
  \defaultfontfeatures[\rmfamily]{Ligatures=TeX,Scale=1}
\fi
% Use upquote if available, for straight quotes in verbatim environments
\IfFileExists{upquote.sty}{\usepackage{upquote}}{}
\IfFileExists{microtype.sty}{% use microtype if available
  \usepackage[]{microtype}
  \UseMicrotypeSet[protrusion]{basicmath} % disable protrusion for tt fonts
}{}
\makeatletter
\@ifundefined{KOMAClassName}{% if non-KOMA class
  \IfFileExists{parskip.sty}{%
    \usepackage{parskip}
  }{% else
    \setlength{\parindent}{0pt}
    \setlength{\parskip}{6pt plus 2pt minus 1pt}}
}{% if KOMA class
  \KOMAoptions{parskip=half}}
\makeatother
\usepackage{xcolor}
\IfFileExists{xurl.sty}{\usepackage{xurl}}{} % add URL line breaks if available
\IfFileExists{bookmark.sty}{\usepackage{bookmark}}{\usepackage{hyperref}}
\hypersetup{
  hidelinks,
  pdfcreator={LaTeX via pandoc}}
\urlstyle{same} % disable monospaced font for URLs
\setlength{\emergencystretch}{3em} % prevent overfull lines
\providecommand{\tightlist}{%
  \setlength{\itemsep}{0pt}\setlength{\parskip}{0pt}}
\setcounter{secnumdepth}{-\maxdimen} % remove section numbering
\ifLuaTeX
  \usepackage{selnolig}  % disable illegal ligatures
\fi

\author{}
\date{}

\begin{document}

In this chapter, we give a general introduction to image classification,
which is one of the basic machine learning problems and we also discuss
some popular datasets: MNIST, CIFAR and ImageNet.

\hypertarget{introduction-to-machine-learning}{%
\section{Introduction to machine
learning}\label{introduction-to-machine-learning}}

Machine learning, a subset of the field of artificial intelligence (AI),
is a field that uses the algorithms to explore the underlying patterns
in the data. It is an interdisciplinary field involving many majors such
as mathematics, statistics and computer science.

Mitchell \cite{mitchell1997machine} gives a formal definition of
learning "A computer program is said to learn from experience \(E\) with
respect to some class of tasks \(T\) and performance measure \(P\), if
its performance at tasks in \(T\), as measured by \(P\), improves with
experience \(E\)." Here the experience \(E\), task \(T\) and performance
measure \(P\) can be chosen from a broad range.

The machine learning algorithms can be broadly categorized into three
divisions based on the approach, the data type and the task to be
solved: supervised learning algorithms, unsupervised learning algorithms
and reinforcement learning algorithms. We will focus on the supervised
learning algorithms here.

A typical supervised machine learning task consists of the dataset, the
model and the learning algorithm. The data can be selected from a wide
variety like digits, pictures, words, etc. The dataset includes the
training set, validation set and test set. For the clarification
problem, we call the dataset involved in training the model the training
set, the dataset used for selecting the optimal model the validation set
and the dataset to be applied with the optimal model to measure the
corresponding generalization ability the test set. The data is used to
help the algorithm to build a mathematical model to extract the patterns
or learn the experience. This process is called training the model. The
main task of the model is to extract the features and characteristics
from the training set for learning and generalize it to the previously
unseen test set for inferring. The generalization ability is defined as
the performance of the model when making inferences on the previously
unseen data.

Supervised learning algorithms try to build a mathematical model of a
set of data that contains both the inputs and the desired outputs
\cite{russell2010artificial}. The inputs, which are also called the
predictors or the independent variables, are used to predict the values
of the outputs, which are also called the responses, the labels or the
dependent variables \cite{friedman2001elements}. If the the range of the
outputs is categorical or has finite numbers, the supervised learning
algorithm is used for classification. If there are infinite many numbers
in the the range of the outputs, the supervised learning algorithm is
used for regression.

The no free lunch theorem \cite{wolpert1996lack} for machine learning
states that, averaged over all possible data generating distributions,
every classification algorithm has the same error rate when classifying
previously unseen examples. It means no machine learning algorithm is
universally any better than any other in some sense
\cite{goodfellow2016deep}. Therefore we need to select the optimal
algorithm based on the specific problem, dataset and task since no model
dominates all the time.

There are many classic supervised learning algorithms that are widely
used in various fields. For example, linear regression, logistic
regression, support vector machine (SVM), K nearest neighbor (KNN) and
so on. For these supervised learning algorithms, the procedure includes
6 steps to be performed \cite{praveena2017literature}.

\begin{enumerate}
\item
  The type of data to be used as the training set needs to be determined
  by the people. For instance, the data can be an image, a word or a
  vector.
\item
  Collect a training set and test set representative of the real-world
  use of the model.
\item
  Resolve the input feature representation of the learned model. For
  example, the input obtained from measurements can be transformed to a
  feature vector while an image can be transformed to a matrix.
\item
  Choose the structure of the learned model and the corresponding
  learning algorithm. This is to determine which model to use and how to
  optimize the model.
\item
  Finish the experiment design and execute the learning algorithm on the
  training set.
\item
  Evaluate the accuracy of the learned model on the test set.
\end{enumerate}

\end{document}
